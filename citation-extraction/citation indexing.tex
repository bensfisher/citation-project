\documentclass[12pt]{article}

%==============Packages & Commands==============
\usepackage{graphicx}
\usepackage{fancyvrb}
\usepackage{tikz}
%%%<
\usepackage{verbatim}
%\usepackage[active,tightpage]{preview}
%\PreviewEnvironment{tikzpicture}
%\setlength\PreviewBorder{5pt}%

\usepackage{geometry}                		% See geometry.pdf to learn the layout options. There are lots.
\geometry{letterpaper}                   		% ... or a4paper or a5paper or ...
%\geometry{landscape}                		% Activat\usetikzlibrary{arrows}e for for rotated page geometry
%\usepackage[parfill]{parskip}    		% Activate to begin paragraphs with an empty line rather than an indent
\usepackage{graphicx}				% Use pdf, png, jpg, or eps§ with pdflatex; use eps in DVI mode
								% TeX will automatically convert eps --> pdf in pdflatex
\usepackage{amssymb}

\usepackage[ruled,vlined]{algorithm2e}
\usetikzlibrary{arrows}
\usepackage{alltt}
\usepackage[T1]{fontenc}
\usepackage[utf8]{inputenc}
\usepackage{indentfirst}
%\usepackage[longnamesfirst]{natbib} % For references
%\bibpunct{(}{)}{;}{a}{}{,} % Reference punctuation
\usepackage{changepage}
\usepackage{setspace}
\usepackage{booktabs} % For tables
\usepackage{rotating} % For sideways tables/figures
\usepackage{amsmath}
\usepackage{multirow}
\usepackage{color}
\usepackage{harvard}
\renewcommand{\harvardurl}{URL: \url}
\usepackage{dcolumn}
\usepackage{comment}
%\usepackage{fullwidth}
\newcolumntype{d}[1]{D{.}{\cdot}{#1}}
\newcolumntype{.}{D{.}{.}{-1}}
\newcolumntype{3}{D{.}{.}{3}}
\newcolumntype{4}{D{.}{.}{4}}
\newcolumntype{5}{D{.}{.}{5}}
\usepackage{float}
\usepackage[hyphens]{url}
%\usepackage[margin = 1.25in]{geometry}
%\usepackage[nolists,figuresfirst]{endfloat} % Figures and tables at the end
\usepackage{subfig}
\captionsetup[subfloat]{position = top, font = normalsize} % For sub-figure captions
\usepackage{fancyhdr}
%\makeatletter
%\def\url@leostyle{%
%  \@ifundefined{selectfont}{\def\UrlFont{\sf}}{\def\UrlFont{\small\ttfamily}}}
%\makeatother
%% Now actually use the newly defined style.
\urlstyle{same}
\usepackage{times}
\usepackage{mathptmx}
%\usepackage[colorlinks = true,
%						bookmarksopen = true,
%						pagebackref = true,
%						linkcolor = black,
%						citecolor = black,
% 					urlcolor = black]{hyperref}
%\usepackage[all]{hypcap}
%\urlstyle{same}
\newcommand{\fnote}[1]{\footnote{\normalsize{#1}}} % 12 pt, double spaced footnotes
\def\citeapos#1{\citeauthor{#1}'s (\citeyear{#1})}
\def\citeaposs#1{\citeauthor{#1}' (\citeyear{#1})}
\newcommand{\bm}[1]{\boldsymbol{#1}} %makes bold math symbols easier
\newcommand{\R}{\textsf{R}\space} %R in textsf font
\newcommand{\netinf}{\texttt{NetInf}\space} %R in textsf font
\newcommand{\iid}{i.i.d} %shorthand for iid
\newcommand{\cites}{{\bf \textcolor{red}{CITES}}} %shorthand for iid
%\usepackage[compact]{titlesec}
%\titlespacing{\section}{0pt}{*0}{*0}
%\titlespacing{\subsection}{0pt}{*0}{*0}
%\titlespacing{\subsubsection}{0pt}{*0}{*0}
%\setlength{\parskip}{0pt}
%\setlength{\parsep}{0pt}
%\setlength{\bibsep}{2pt}
%\renewcommand{\headrulewidth}{0pt}

%\renewcommand{\figureplace}{ % This places [Insert Table X here] and [Insert Figure Y here] in the text
%\begin{center}
%[Insert \figurename~\thepostfig\ here]
%\end{center}}
%\renewcommand{\tableplace}{%
%\begin{center}
%[Insert \tablename~\theposttbl\ here]
%\end{center}}

\newcommand\independent{\protect\mathpalette{\protect\independenT}{\perp}}
\def\independenT#1#2{\mathrel{\rlap{$#1#2$}\mkern2mu{#1#2}}}
\newcommand{\N}{\mathcal{N}}
\newcommand{\Y}{\bm{\mathcal{Y}}}
\newcommand{\bZ}{\bm{Z}}

\usepackage[colorlinks = TRUE, urlcolor = black, linkcolor = black, citecolor = black, pdfstartview = FitV]{hyperref}


%============Article Title, Authors==================
\title{\vspace{-2cm} Automated Citation Indexing of Government Documents} 


\author{ Ben Fisher \and Bruce Desmarais} \date{\today}



%===================Startup=======================
\begin{document}
\maketitle

%=============Abstract & Keywords==================

\begin{abstract}

 \noindent  We extract citations from government documents. \\~\\

\end{abstract}

Intended Journals \begin{itemize}
\item {\em Journal of Informetrics}
\item {\em Scientometrics}
\item {\em PLoS ONE altmetrics collection}
\end{itemize}

\thispagestyle{empty}
\newpage
\doublespacing


\section{Introduction}
% write this as if the main goal is to extract scientific cites, with other references as an extra benefit
Assessing the impact of scientific research in applied contexts is challenging, but necessary in evaluating the returns to society from investing in basic science. The measures used to assess the scientific importance of basic research are largely derived from citation in the bibliographies of scientific papers. One of the major challenges in assessing impact outside of the scientific community is that applied users of basic science rarely produce systematic and publicly available bibliographies. One exception is the recent focus on patents \cite{huang2014,liaw2014,huang2015,wong2015}. One area of real-world impact of science that is highly relevant to all members of society is the use of science in the creation of government policy \cite{nrc2012}. Indeed, the technical basis of government regulations has grown so salient in recent decades that regulations are regularly challenged in court on the basis of the science underpinning major provisions \cite{ellig2013,morrall2014}. As such, while developing policies, government officials regularly create publicly available and heavily vetted documents in which they present the research on which a regulation is based. If we can systematically extract and organize this information, we can directly assess the impact of scientific research on public policy.

% We need to do this automatically
Our goal is to assess the degree to which a publication in a peer-reviewed scientific journal has been directly used in benefit-cost analyses conducted by policymakers. These benefit-cost analyses, termed ``Regulatory Impact Analyses'' (RIAs) in the US federal government, directly inform the process of crafting government regulations. It is now standard procedure in dozens of national governments, and many sub-national governments such as the US states, for policymakers to conduct a benefit-cost analysis to inform the provisions in government regulations \cite{hahn2007}. By indexing these analyses for citations to the scientific literature, we construct a measure of the degree to which an individual paper or journal, has directly informed the process of crafting government regulations. We build upon a dataset introduced by \cite{desmarais2014} in which citations from 104 RIAs covering major regulations passed by the US federal government during the period 2008--2012. Though feasible for small collections of benefit-cost analyses, hand-indexing would be prohibitively expensive for larger corpora, which could easily number in the tens of thousands if they covered longer spans of time, other nations, or subnational governments. We use the dataset introduced by \cite{desmarais2014} to train and evaluate procedures for automating citation indexing for RIAs. We then apply automated indexing to an expanded corpus of RIAs from the US federal government, and use these citations to construct a regulatory policy impact factor.

\subsection{Tasks}

\begin{itemize}
\item Write ``outline of the paper'' paragraph (BD)


\end{itemize}

\section{Literature Review}
The development of citation indexing has primarily focused on indexing scholarly citations in the confines of large-scale projects like \citeasnoun{garfield1964}'s Science Citation Index, which eventually morphed present Web of Science or \citeasnoun{lawrenceetal1999}'s Citeseer database. The process by which proprietary services like Web of Science or Google Scholar actually extract citations is opaque. The Citeseer project employs a heuristic, rule-based approach to identify citation strings within text. For example, the ParsCit parser developed by \citeasnoun{councilletal2008}, which Citeseer currently uses, identifies the bibliography of a paper by searching for ``bibliography'' or synonyms like ``references''. If one of these is found past the first 40\% of the paper, then the parser extracts all of the text past that line through the end of the paper or the appearance of another section heading, such as ``appendix''. The citation strings are then separated using regular expressions. 

There has been limited publicly available work done on improving automated citation indexing beyond the Citeseer project. Papers on citation metadata extraction generally rely on citation datasets downloaded from Citeseer or manually compiled datasets, rather than developing new methods for parsing documents themselves \cite{anzaroot2013}. Additionally, the automated citation indexing methods used by Citeseer and similar databases is tailored for extracting scholarly citations, making the application of existing algorithms problematic for extracting heterogeneous citations from government reports.

Citation indexing can be seen as a special case of token tagging in text (i.e., part of speech (POS)) tagging. Consider a document in which every word (or token) has a tag from a given tag set (e.g., POS = \{noun, verb, adjective,...,\}). The word ``run'', for example, would likely be tagged as a verb, but could also be tagged as a noun (e.g., ``I am going on a run''). Citation indexing could be seen as a two-part token tagging task. First, within a document, we need to tag tokens regarding whether they are within a citation instance, with adjacent sets of tokens in a citation instance representing a single citation. Second, after extracting citations, tagging the tokens within the citation index as title, author, etc. For our purposes, we really only need a tagger that performs well at the first task, as the second task can be performed by hand or with the assistance of other citation indexes (e.g., Google Scholar). The Stanford POS tagger can be used to train a tagger using a custom tag set \cite{toutanova2003}. We could use a grep for scientific cite titles in our hand-coded data to tag tokens as starting a cite title, within a cite title, ending a cite title and not in a cite title. Other tags may be useful, but whatever the tag set, we could train the Stanford Tagger and test extraction rules for automatically indexing cites.  


\subsection{Tasks}

\begin{itemize}
\item Review "altmetrics" lit (BD)
\item  read huang2014 and Liaw2014 to develop notes about framing and presenting the contribution (BF)
\end{itemize}

\section{RIA Description}

\section{Research Design}
We use 104 RIAs whose citations have already been extracted by hand as a test sample. The structure of these RIAs is very heterogeneous. They vary in length from a few dozen pages to hundreds of pages. References may appear in a reference section, in footnotes, or be scattered throughout the text. The citations of interest include both scholarly citations and non-scholarly citations, such as existing federal regulations, court cases, and treaties. \\ 

To assess how our own approach compares to existing citation indexing programs, we run the ParsCit program on the manually coded sample of RIAs. ParsCit was originally developed for indexing scholarly citations from journal articles for the Citeseer project \cite{councilletal2008}. In addition to extracting citations, the program also extracts citation metadata, such as author, article title, and journal. We do not expect it to handle metadata extraction for non-scholarly citations. However, it may be able to extract text containing non-scholarly citations, which could then be manually coded. This would greatly speed up the current coding process, which requires research assistants to read the entire document for citations. If ParsCit performs well enough at extracting all citation types, then there is no need to develop new indexing software for this project.\\

We use four different string distance metrics to measure similarity between the manually collected citations and the citations collected by ParsCit: Levenshtein, Jaccard, Sorenson, and partial string similarity. Levenshtein distance is measured by calculating the number of insertions, deletions, or substitutions needed to turn one string into another string [CITE Levenshtein 1966]. To calculate the Jaccard distance, the number of characters shared by two strings is divided by the total number of characters and subtracted from 1 [CITE Jaccard 1901]. The Sorenson distance measure is very similar to the Jaccard measure. Instead of looking at the number of shared characters, the Sorenson measure calculates the number of shared bigrams. This number is multiplied by 2 and then divided by the total number of bigrams in each string and subtracted from 1 [CITE Sorenson 1948]. Except for Levenshtein distance, all measures fall on a scale of 0 to 1, with 0 indicating an exact match and 1 indicating two completely different strings. The Levenshtein measure is normalized to the same scale for the purposes of comparison. The final measure, partial string similarity, uses substring matching and measures the percentage of characters in string $a$ that appear in string $b$.\\

\subsection{Tasks}

\begin{enumerate}
\item How did we OCR the RIAs in order to read them into ParsCit (BF)?
\item See which distance metrics perfectly predict a match at a given distance (BF)
\item Assess the cost of hand-coding our test data (BF)
\item Assessing the precision and recall of parscit (BF)
\item See how this changes over the RIA date (BF)
\end{enumerate}



\section{Results}

\subsection{Tasks}

\begin{itemize}
\item Meet with Lee's student (Kyle?) to see how we can improve
\end{itemize}


\section{Conclusion}

\subsection{Tasks}

\begin{itemize}
\item Why the world needs this? (BD)
\end{itemize}


\clearpage
\singlespace
\bibliographystyle{apsr}
\nocite{*}
\bibliography{citationextraction}
\clearpage

\end{document}




























\documentclass[12pt]{article}
\usepackage[swedish,icelandic,english]{babel}
\usepackage[T1]{fontenc}
\usepackage{afterpage}
\usepackage{setspace}
\usepackage{amssymb}
\usepackage{epsfig}
\usepackage{dcolumn}
\usepackage{afterpage}
\usepackage{multirow}
\usepackage{booktabs}
\usepackage{colortbl}
\usepackage{fullpage}
\usepackage{times}
\usepackage{amsthm}
\usepackage{amsfonts}
\usepackage{lscape}
\usepackage{threeparttable}
\usepackage{psfrag}
\usepackage{dcolumn}
\usepackage{fleqn}
\usepackage{epic}
\usepackage{longtable}
\usepackage{rotating}
\usepackage{latexsym}
\usepackage{url}
\usepackage{endnotes}
\usepackage{url}
\usepackage{subcaption}
\usepackage{subfigure}
\usepackage [abbr] {harvard}
\usepackage[hang,flushmargin]{footmisc} 
%\usepackage{natbib}
\renewcommand{\harvardurl}{URL: \url}
\usepackage{amsmath, bm}
\usepackage{afterpage}
\usepackage{graphics}
\usepackage{graphicx}
\usepackage{paralist} % \setdefaultenum{(a)}{i}{A}{1}
\setlength{\LTcapwidth}{6in}

\def\p3s{\phantom{xxx}}
\parskip=0pt
\parindent=0pt
\interfootnotelinepenalty=10000
\begin{document}
\parindent=30pt
\renewcommand{\thefootnote}{\fnsymbol{footnote}}



\doublespace
\setcounter{page}{1}
\renewcommand{\thefootnote}{\fnsymbol{footnote}}

\normalsize
\hyphenpenalty 1000

\singlespace
\begin{center}

\title{\vspace{-2cm} Automated Citation Indexing of Government Documents} \\


\vspace{1cm}
\large
\normalsize\vspace*{8.5cm}


%\textsc{}\\
%\textit{}\\

\end{center}
\vspace{3.5cm}

%\begin{center}
%\large \textbf{ABSTRACT}
%\end{center}

\singlespace \noindent 

% \noindent Abstract

\newpage
\doublespace
\setcounter{page}{1}
\renewcommand{\thefootnote}{\arabic{footnote}} \setcounter{footnote}{0}
\normalsize
\hyphenpenalty 1000


